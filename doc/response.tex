\documentclass[12pt]{article}
\usepackage[utf8]{inputenc}

\usepackage{color}
\usepackage{graphicx}

\usepackage{xspace}

\usepackage{lmodern}
\usepackage{amssymb,amsmath}

\usepackage[pdfencoding=auto, psdextra]{hyperref}

\usepackage{natbib}
\bibliographystyle{chicago}

\newcommand{\eref}[1]{Eq.~(\ref{eq:#1})}
\newcommand{\fref}[1]{Fig.~\ref{fig:#1}}

\newcommand{\Rx}[1]{\ensuremath{{\mathcal R}_{#1}}} 
\newcommand{\Ro}{\Rx{0}}
\newcommand{\Rc}{\Rx{\textrm{\tiny{contact}}}}
\newcommand{\RR}{\ensuremath{{\mathcal R}}}
\newcommand{\Rhat}{\ensuremath{{\hat\RR}}}
\newcommand{\tsub}[2]{#1_{{\textrm{\tiny #2}}}}

\newcommand{\dd}[1]{\ensuremath{\, \mathrm{d}#1}}
\newcommand{\dtau}{\dd{\tau}}
\newcommand{\dx}{\dd{x}}
\newcommand{\dsigma}{\dd{\sigma}}

\newcommand{\rev}{\subsection*}
\newcommand{\revtext}{\textsf}
\setlength{\parskip}{\baselineskip}
\setlength{\parindent}{0em}

\newcommand{\comment}[3]{\textcolor{#1}{\textbf{[#2: }\textsl{#3}\textbf{]}}}
\newcommand{\jd}[1]{\comment{cyan}{JD}{#1}}
\newcommand{\swp}[1]{\comment{magenta}{SWP}{#1}}
\newcommand{\dc}[1]{\comment{blue}{DC}{#1}}
\newcommand{\jsw}[1]{\comment{green}{JSW}{#1}}
\newcommand{\hotcomment}[1]{\comment{red}{HOT}{#1}}

\newcommand{\psymp}{\ensuremath{p}} %% primary symptom time
\newcommand{\ssymp}{\ensuremath{s}} %% secondary symptom time
\newcommand{\pinf}{\ensuremath{\alpha_1}} %% primary infection time
\newcommand{\sinf}{\ensuremath{\alpha_2}} %% secondary infection time

\newcommand{\psize}{{\mathcal P}} %% primary cohort size
\newcommand{\ssize}{{\mathcal S}} %% secondary cohort size

\newcommand{\gtime}{\tau_{\rm g}} %% generation interval
\newcommand{\gdist}{g} %% generation-interval distribution
\newcommand{\idist}{\ell} %% incubation period distribution

\newcommand{\total}{{\mathcal T}} %% total number of serial intervals

\usepackage{lineno}
\linenumbers

\begin{document}

\noindent Dear Editor:

Thank you for the chance to revise and resubmit our manuscript.
We have tried to highlight the limitation of our correlation analysis and avoid linking its results to any causal statements.
We have also tried to clarify that the causal conclusions are supported by the modeling analysis, rather than the correlation analysis.
Below please find our detailed responses to reviewers.

\revtext{Reviewer \#1 Comments for the Author:}

\revtext{The authors have done a good job responding to my previous set of comments and have made appropriate revisions to their manuscript.}

Thank you for your review.

\revtext{Examining the updated timeseries results, my remaining concern regards the use of Pearson correlation of weekly case incidence in the analysis of dynamic coupling between community and campus, which features strongly in the significance statement and generally as a contribution of the paper.}

\revtext{I'm not sure it's appropriate to give p values based on a null model of independent identically distributed samples, when timeseries data like this is autocorrelated and non-stationary. If possible this should be reviewed by someone with a background in analysis of coupled timeseries data. At minimum, the methods used to compute correlation should be given in more detail and the text should be examined to ensure that causal inference is not being overstated.}

Thank you for pointing this out.
We recognize that the causal language has been too strong.
In particular, the causal inference is primarily supported by the mathematical modeling analysis, rather than the correlation analysis.
We have revised the paper throughout to make the distinction clear and not overstate the implications of the correlations.
Overall, we have tried to avoid linking correlations to causal statements.

We have also added the following paragraph in the beginning of the results section to make the limitation of the method clearer:

``
In order to compare the transmission dynamics on PU campus and in nearby communities, we calculate the Pearson correlation between the weekly numbers of cases from PU and those from nearby communities.
Confidence intervals are calculated using Fisher's Z transformation, and the corresponding significance is calculated using asymptomatic t approximation \citep{hollander2013nonparametric}.
The standard method of calculating Perason correlations and the associated significance assumes that the samples are independent and identically distributed---these assumptions generally do not hold for epidemic time series data, which are autocorrelated and non-stationary.
Nonetheless, the Pearson correlation has been the standard measure in epidemic time series analyses and can provide useful information about the strength of spatial coupling \citep{rohani1999opposite,grenfell2001travelling,viboud2006synchrony,baker2019epidemic}.
We note correlation analyses do not provide direct evidence for causality and therefore should be interpreted with care---we later investigate the role of community transmission using a mathematical model.''

\revtext{For example, it is stated that: "we are not able to infer the direction of causality—", but I am concerned this may be an understatement of the limitations with respect to making causal inference of any kind from the correlation results.}

Thank you for pointing this out.
We recognize that our phrasing was unclear. 
This sentence was meant to point at the limitation of our modeling analysis, rather than the correlation analysis.
We have tried to highlight the limitation of our correlation results clearer while making sure that we don't conflate conclusions from correlation analyses and modeling analyses:

``There are several limitations to our analysis.
While we demonstrate strong spatiotemporal correlation in the spread of SARS-CoV-2, correlations do not provide causal link for community transmission.
For example, even if there was no mixing between campus and community populations, two populations can exhibit correlated epidemic dynamics if both populations are changing behavior in similar ways, reflecting changes in intervention measures in the greater New Jersey area.
However, it seems biologically implausible that there would be no mixing between the two populatins---other studies have also identified community transmission to be an important source of SARS-CoV-2 infections on campus \citep{fox2021response,hamer2021assessment}.
Our mathematical modeling illustrates that community transmission can explain the observed dynamics on PU campus, but we are not able to infer the direction of causality''

\revtext{Reviewer \#2 Comments for the Author:}

\revtext{Thank you for the detailed point-by-point response from the authors. I think the revisions really help with the clarity of the manuscript and allow for a more reliable/realistic comparison between community and campus dynamics. I also appreciate the extended sensitivity analysis that accounts for some of the additional epidemiological changes during Omicron. I am happy to recommend this work for publication.}

Thank you for your review.

\bibliography{university-covid}

\end{document}
